\subsection{Определение случайного вектора}

Можно дать два эквивалентных определения случайному вектору. Первое определение:

\begin{definition}[случайый вектор]
    Пусть $\xi_1, \ldots, \xi_n$ --- случайные величины, тогда $(\xi_1, \ldots, \xi_n)$ --- случайный вектор.
\end{definition}

Для введения второго определения рассмотрим понятие измеримости функций.

\begin{definition}[борелевская $\sigma$-алгебра (борелеское множество)]
    Это множества на числовой прямой (лучи, отрезки, точки), принадлежащие минимальной $\sigma$-алгебре над совокупностью всех сегментов $[a,b]$.
\end{definition}

Про сигма-алгебры можно почитать \href{https://tvims.nsu.ru/chernova/tv/lec/node9.html}{здесь}.

\begin{definition}[измеримость]
    Функция является $(\mathbb{R}, \mathcal{B}(R))$-измеримой, если 
    \begin{math}
        \forall B \in \mathcal{B} (\mathbb{R} ) \; f^{-1}(B) \in \mathcal{F}.
    \end{math}
\end{definition}

Второе определение:

\begin{definition}[случайный вектор]
    Если $\xi: \Omega \rightarrow \mathbb{R}^k$ является $(\mathcal{F} | \mathcal{B} (\mathbb{R} ^k))$-измеримой функцией, то $\xi$ называется случайным вектором. 
\end{definition}

\subsection{Функции от случайного вектора}

\begin{definition}[борелевская функция]
    Числовая функция $f: \mathbb{R}^n \rightarrow \mathbb{R}^k$, заданная на прямой, называется \textit{борелевской}, если прообраз каждого борелевского множества есть борелевское множество, т. е. $f$ --- $(\mathcal{B} (\mathbb{R}^n | \mathcal{B} (\mathbb{R}^k))$-измеримая.
\end{definition}

Любая непрерывная функция является борелевской.

Следствие: $\xi+\eta$, $\xi \cdot \eta$, $\xi-\eta$, $\frac{\xi}{\eta}$ --- случайные величины.

\subsection{Функция распределения случайного вектора}

\begin{definition}[распределение случайного вектора]
    Распределением случайного вектора $\xi$ называется функция 
    \begin{equation}
        P_\xi : \mathcal{B} (\mathbb{R}^n) \rightarrow [0,1]
    \end{equation}
\end{definition}

\subsection{Характеристическая функция}

\subsection{Гауссовский вектор}

\subsection{Многомерная ЦПТ}

\newpage