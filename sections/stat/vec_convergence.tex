\begin{definition}[случайный вектор]
    Пусть $(\Omega, \mathcal{F}, P)$ --- вероятностное пространство. 
    Тогда отображение $\xi : \Omega \to \mathbb{R}^n$ называется случайным вектором, если
    \begin{equation}
        \forall B \in \mathcal{B} (\mathbb{R}^n) \; \; \; \; \; \xi^{-1} (B) \in \mathcal{F}
    \end{equation}
Другими словами, функция $\xi$ измерима, то есть, те $\omega$, которые порождают соответствующие значения элементов вектора лежат в сигма-алгебре на $\Omega$.
\end{definition}

\begin{definition}[верятностная мера, порожденная вектором с.в.]
    Вероятностная мера $P_\xi$, заданная на $(\mathbb{R}^n, \mathcal{B}(\mathbb{R}^n))$-измеримом пространстве по следующему правилу:
    \begin{equation}
        P_\xi(B) = P(\xi \in B) \; \; \; \; \; \forall B \in \mathcal{B}(\mathbb{R}^n)
    \end{equation}
    называется вероятностной мерой, порожденной вектором $\xi$.
\end{definition}

\subsection{Вероятностно-статистическая модель}

Известно, что $P_\xi \in \mathcal{P}$ --- принадлежит классу вероятностных законов (класс --- произвольная совокупность множеств, обладающих определённым свойством или признаком).

\subsection{Виды сходимостей случайных векторов}

\subsection{Теорема о наследовании сходимостей, лемма Слуцкого}

\subsection{Дельта-метод}

\subsection{Многомерный дельта-метод}

\newpage