\documentclass{article}
 
%Russian-specific packages
%--------------------------------------
\usepackage[T2A]{fontenc}
\usepackage[utf8]{inputenc}
\usepackage[russian]{babel}
%--------------------------------------
 
\usepackage{amsthm}
\usepackage{amssymb}

\newtheoremstyle{mydef}
{\topsep}{\topsep}%
{}{}%
{}{}
{\newline}
{%
  \rule{\textwidth}{0.1pt}\\*%
  \thmname{#1}~\thmnumber{#2}\thmnote{\ -\ #3}.\\*[-1.5ex]%
  \rule{\textwidth}{0.1pt}}%

\theoremstyle{mydef}
\newtheorem{definition}{Определение}

\newtheoremstyle{myth}
{\topsep}{\topsep}%
{}{}%
{\bfseries}{}
{\newline}
{%
  \rule{\textwidth}{0.1pt}\\*%
  \thmname{#1}~\thmnumber{#2}\thmnote{\ -\ #3}.\\*[-1.5ex]%
  \rule{\textwidth}{0.1pt}}%

\theoremstyle{myth}
\newtheorem{theorem}{Теорема}

\usepackage{graphicx}

%Hyphenation rules
%--------------------------------------
\usepackage{hyphenat}
\hyphenation{ма-те-ма-ти-ка вос-ста-нав-ли-вать}
%--------------------------------------
 
\begin{document}

\begin{abstract}
    Конспект конспекта лекций курса Теории Вероятностей онлайн-магистратуры МФТИ по современной комбинаторике. \newline
    Лектор: Райгородский Андрей Михайлович, Жуковский Максим Евгеньевич.
\end{abstract}

\tableofcontents
 
\newpage

\section{Комбинаторика}

\section{Классическая вероятность}

Для классической вероятности характерны следующие свойства:
\begin{enumerate}
    \item исходы образуют полную группу событий, т.е. хотя бы один из возможных элементарных исходов произойдет;
    \item события попарно несовместны --- может произойти только одно из них, например, может выпасть только 1 из граней кубика;
    \item все исходы равновероятны.
\end{enumerate}

\subsection{Случайное событие, вероятность}

\begin{definition}[Вероятность]
    Вероятность --- отношение числа благоприятствующих исследуемому событию исходов к числу всех возможных исходов.
\end{definition}

\begin{definition}[Событие]
    Событие A --- подмножество множества элементарных исходов, из которых складывается искомое.
\end{definition}

\begin{definition}[Пространство элементарных исходов]
    Пространство элементарных исходов $\Omega$ --- пространство всех возможных исходов эксперимента.
\end{definition}

\textbf{Свойства вероятности}:
\begin{enumerate}
    \item $P(\Omega) = 1$ --- вероятность того, что произойдет любой из возможных исходов равна 1 по св-ву классической вероятности;
    \item $P(\emptyset) = 0$, $P(A) = 0 \Leftrightarrow  A = \emptyset$;
    \item $P(A \sqcup B) = P(A) + P(B)$ --- сигма-аддитивность, распространяется и на случай конечного объединения событий;
    \item $P(A \cup B) = P(A) + B(B) - P(A \cup B)$;
    \item $P(A_1 \cup A_2 \cup \ldots A_n) = P(A_1) + P(A_2) + \ldots + P(A_n) - P(A_1 \cap A_2) - \ldots - P(A_{n-1} \cap A_n) + \ldots + (-1)^{n-1} P(A_1 \cap \ldots \cap A_n)$ --- формула включений-исключений;
    \item $\bar{A} := \Omega \setminus A$; $P(\bar{A}) = 1 - P(A)$ --- отрицание события; 
    \item $P(A_1 \cup A_2 \cup \ldots A_n) \leq P(A_1) + \ldots + P(A_n)$;
    \item для $A_{i+1} \subseteq A_i$ таких, что $\bigcap_{i=1}^{\infty} A_i = \emptyset$, то $\lim_{k \to \infty} P(A_i) = 0$ --- свойство непрерывности. 
\end{enumerate}

\subsection{Условная вероятность}
 
\begin{definition}[Условная вероятность]
    Условная вероятность --- это вероятность того, что событие A произойдет, при условии, что событие B уже произошло. \\
    Обозначается $P(A|B)$ --- <<вероятность A при условии B>>. \\
    \begin{equation}
        P(A|B) = \frac{|A\cup B|}{|B|} = \frac{P(A \cup B)}{P(B)}
    \end{equation}
\end{definition}

\subsubsection{Теорема умножения и независимость событий}

Теорема умножения позволяет избавиться от нуля в знаменателе, в формуле условной вероятности.

\begin{equation}
    P(A|B) \cdot P(B) = P(A \cup B)
\end{equation}

Событие A не зависит от события B, если $P(A|B) = P(A)$, т.е. наступление события B никак не влияет на вероятность наступления события A. \\
Если событие A независимо от события B, то имеет место равенство:

\begin{equation}
    P(A) \cdot P(B|A) = P(B) \cdot P(A)
\end{equation}

Отсюда, при $P(A)>0$ находим, что 

\begin{equation}
    P(A|B) = P(B)
\end{equation}

т.е. событие B также независимо от A. Т.о. свойство независимости событий взаимно.

События $A_1, \ldots, A_n$ могут быть независимыми в совокупности и попарно. 

\subsubsection{Формула полной вероятности}

Разделим множество $\Omega$ на непересекающиеся подмножества $B_1, \ldots, B_k$.

\subsubsection{Формула Байеса}

\section{Схема испытаний Бернулли}

\section{Случайные величины}

\subsection{Вероятностные пространства в классическом случае и в схеме испытаний Бернулли}

Обобщим вероятностные пространства для классической вероятности и схемы испытаний Бернулли. \\
Вероятность элементарного исхода $\omega_i$ будем обозначать через $p_i$. \\
Очевидно, что $p_i \in [0,1]$, и $p_1 + \ldots + p_n = 1$. \\
Остальные свойства вероятности так же выполнены для обобщенного понятия вероятности.

% Please add the following required packages to your document preamble:
% \usepackage{graphicx}
\begin{table}[h]
    \centering
    \resizebox{\textwidth}{!}{%
    \begin{tabular}{|l|l|l|l|}
    \hline
    \multicolumn{1}{|c|}{Понятие}                                                  & \multicolumn{1}{c|}{Классическая вероятность} & \multicolumn{1}{c|}{Сх. испытаний Бернулли}                          & \multicolumn{1}{c|}{Обобщение}            \\ \hline
    \begin{tabular}[c]{@{}l@{}}Пространство\\ элементарных \\ исходов\end{tabular} & $\Omega = \{\omega_1, \ldots, \omega_n\}$     & $\Omega = \{\omega_1, \ldots, \omega_{2^n}\}$                        & $\Omega = \{\omega_1, \ldots, \omega_n\}$ \\ \hline
    \begin{tabular}[c]{@{}l@{}}Элементарный \\ исход\end{tabular}                  & $\omega_i$ -- 1 из возможных исходов          & $\omega_i = (x_1, \ldots, x_n)$, где $x_i=\{0,1\}$                   & $\omega_i$ -- элементарный исход          \\ \hline
    \begin{tabular}[c]{@{}l@{}}Вероятность\\ элементарного \\ исхода\end{tabular}  & $P(\omega_i) = \frac{1}{n}$                   & $P(\omega_i) = p^{\sum_{i=1}^n x_i}\cdot (1-p)^{n-\sum_{i=1}^n x_i}$ & $P(\omega_i) = p_i$, где $p_i \in [0,1]$  \\ \hline
    \begin{tabular}[c]{@{}l@{}}Вероятность \\ события\end{tabular}                 & $P(A) = \frac{|A|}{n}$                        & $P(A) = \sum_{\omega_i \in A} P(\omega_i)$                           & $P(A) = \sum_{\omega_i \in A} p_i$        \\ \hline
    \end{tabular}%
    }
\end{table}

\subsection{Определение случайной величины}

Пусть дано Пространство элементарных исходов $\Omega$, и вероятностная мера $P$ на нём.

\begin{definition}[Случайная величина]
    Случайной величиной называется любая функция $\xi \; : \; \Omega \rightarrow \mathbb{R}$. \\
    Например, для игральной кости $\Omega = \{1,2,3,4,5,6\}$, зададим $\xi$ как $\xi(\omega) = \omega^2$.
\end{definition}

Случайная величина ставит в соответствие каждому элементарному исходу какое-либо конкретное число из множества действительных чисел. \\
Таким образом, "выпадение" какого-то из исходов случайно, но значение случайной величины уже конкретно.

\textbf{Пример.} $\xi =$ $\#$ треугольников в случайном графе. \\
Количество событий, когда на имеющихся вершинах появился треугольник случайно, но, зная количество таких событий, мы можем получить конкретное $\xi$.

\subsection{Функция распределения случайной величины}

Существует важная задача создания методов изучения случайных величин. \\
Случайная величина под влиянием случайных обстоятельств может принимать различные значения. Заранее предсказать, какое значение примет величина невозможно, т.к. оно меняется случайным образом от испытания к испытанию. \\
Для того, чтобы задавать вероятности значений случайных величин вводится понятие \textbf{функции распределения случайной величины}.

\begin{definition}[Распределение случайной величины]
    Распределением случайной величины называется вероятность того, что случайная величина $\xi$ принимает конкретное значение $y_i$:
    \begin{equation}
        P(\xi = y_i) = P(\{\omega_j \; : \; \xi(\omega_j) = y_i\})
    \end{equation}
Т. е. это вероятность всех элементарных исходов, при которых случайная величина принимает значение $y_i$.
\end{definition}

\begin{definition}[Функция распределения случайной величины]
Пусть $\xi$ --- случайная величина, $x \in \mathbb{R}$ --- произвольное действительное число. Вероятность того, что $\xi$ примет значение, меньшее, чем $x$ называется \textit{функцией распределения случайной величины}.

\begin{equation}
    F_\xi (x) := P(\xi \leq x) = \sum_{i \; : \; y_i \leq x} P(\xi=y_i)
\end{equation}

\end{definition}

\section{Математическое ожидание и дисперсия}

\subsection{Математическое ожидание}

Математическое ожидание даёт оценку случайной величины <<в среднем>>, что часто бывает нужно в реальных задачах.

\begin{definition}[Математическое ожидание]
    Пусть $x_1, x_2, \ldots, x_n, \ldots$ --- возможные значения случайной величины $\xi$, а $p_1, p_2, \ldots, p_n, \ldots$ --- соответствующие им вероятности. \\
    Если ряд     
    \begin{equation}
        \mathbb{E} \xi = \sum_{n=1}^\infty x_n p_n   
    \end{equation}
    
    сходится абсолютно, то его сумма называется \textit{математическим ожиданием дискретной случайной величины}.

    Для непрерывных с.в.:
    \begin{equation}
        \mathbb{E}\xi = \int x \cdot p(x) dx = \int x \cdot d F(x)
    \end{equation}

    \textbf{Мат. ожидание линейно:}

    \begin{equation}
        \mathbb{E}(c_1 \xi_1 + c_2 \xi_2) = c_1 \mathbb{E} \xi_1 + c_2 \mathbb{E} \xi_2 
    \end{equation}
\end{definition}

\subsection{Неравенство Маркова}

\begin{theorem}[(Маркова)]
    Пусть $\xi$ принимает только неотрицательные значения. Пусть $a > 0$, тогда

\begin{equation}
    P(\xi \geq a) \leq \frac{\mathbb{E}\xi}{a}
\end{equation}
\end{theorem}

\subsection{Дисперсия}

Важное значение имеет то, насколько сильно случайные величины могут отличаться от своего математического ожидания. Для такой оценки можно использовать дисперсию и некоторые другие числовые характеристики.

\begin{definition}[Дисперсия]
    Дисперсия случайной величины:
    \begin{equation}
        \mathbb{D} \xi := \mathbb{E}(\xi - \mathbb{E})^2 = \mathbb{E} \xi^2 - (\mathbb{E}\xi)^2
    \end{equation}
\end{definition}

\subsection{Неравенство Чебышёва}

\begin{theorem}[Неравенство Чебышёва]
    Пусть $\eta$ --- любая случайная величина. Пусть $b>0$. Тогда:
    \begin{equation}
        P(|\eta - \mathbb{E} \eta| \geq b) \leq \frac{\mathbb{D}\eta}{b^2}    
    \end{equation}
\end{theorem}

\subsection{Абсолютно непрерывные случайные величины}

\section{Независимые случайные величины и закон больших чисел}

\section{Предельные теоремы}

\section{Геометрическая вероятность}

\section{Колмогоровская аксиоматика}

\end{document}