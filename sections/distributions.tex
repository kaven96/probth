В разделе \ref{sec:rand_var} речь шла только о дискретных с.в. Для дальнейшего рассмотрения видов распределений случайных величин введём понятие абсолютно непрерывной случайной величины.

\subsection{Абсолютно непрерывные случайные величины}

Для а.н.с.в. характерна непрерывная функция распределения. Пусть $p(x)$ --- полотность распределения, тогда она обладает свойствами:
\begin{enumerate}
    \item $\forall x \; p(x) \geq 0$;
    \item $\int_{-\infty}^{+\infty} p(x) d(x) =1$;
\end{enumerate}

В а.н. случае функция распределения представляется в виде:
\begin{equation}
    F_\xi(x) = \int_{-\infty}^x p(t) dt
\end{equation}

\subsection{Распределения дискретных случайных величин}

Примеры распределений и их функции распределения можно найти \href{https://tvims.nsu.ru/chernova/tv/lec/node26.html}{на этом сайте}.

\subsection{Распределения а.н.с.в.}

Примеры распределений и их функции распределения можно найти \href{https://tvims.nsu.ru/chernova/tv/lec/node27.html}{на этом сайте}.

\subsection{Сингулярные распределения}

\newpage