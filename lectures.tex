\documentclass{book}
 
%Russian-specific packages
%--------------------------------------
\usepackage[T2A]{fontenc}
\usepackage[utf8]{inputenc}
\usepackage[russian]{babel}
%--------------------------------------
 
%Packages for math input
%--------------------------------------
\usepackage{amsthm}
\usepackage{amssymb}
\usepackage{graphicx}
%--------------------------------------

%Style for theorems and definition environments
%--------------------------------------
%definition
\newtheoremstyle{mydef}
{\topsep}{\topsep}%
{}{}%
{}{}
{\newline}
{%
  \rule{\textwidth}{0.1pt}\\*%
  \thmname{#1}~\thmnumber{#2}\thmnote{\ -\ #3}.\\*[-1.5ex]%
  \rule{\textwidth}{0.1pt}}%

\theoremstyle{mydef}
\newtheorem{definition}{Определение}
%theorem
\newtheoremstyle{myth}
{\topsep}{\topsep}%
{}{}%
{\bfseries}{}
{\newline}
{%
  \rule{\textwidth}{0.1pt}\\*%
  \thmname{#1}~\thmnumber{#2}\thmnote{\ -\ #3}.\\*[-1.5ex]%
  \rule{\textwidth}{0.1pt}}%

\theoremstyle{myth}
\newtheorem{theorem}{Теорема}
%--------------------------------------

%Hyperlinks in text
%--------------------------------------
\usepackage{hyperref}
%--------------------------------------

%Hyphenation rules
%--------------------------------------
\usepackage{hyphenat}
\hyphenation{ма-те-ма-ти-ка вос-ста-нав-ли-вать}
%--------------------------------------
 
%Splitting document 
%--------------------------------------
\usepackage{import}
%--------------------------------------


\begin{document}

% \begin{abstract}
%     Конспект конспекта лекций курса Теории Вероятностей онлайн-магистратуры МФТИ по современной комбинаторике. \newline
%     Лекторы: Райгородский Андрей Михайлович, Жуковский Максим Евгеньевич.
% \end{abstract}

\tableofcontents
 
\newpage

\section{Условные обозначения}

\import{sections/probth/}{symbol_ref.tex}

\chapter{Комбинаторика}

\import{sections/probth/}{combinatorics.tex}

\chapter{Теория вероятностей}

\section{Классическая вероятность}

\import{sections/probth/}{classical_probth.tex}

\section{Схема испытаний Бернулли}

\import{sections/probth/}{bernoulli_sch.tex}

\section{Случайные величины}\label{sec:rand_var}

\import{sections/probth/}{rand_var.tex}

\section{Математическое ожидание и дисперсия}

\import{sections/probth/}{expect_varian.tex}

\section{Распределения случайных величин}

\import{sections/probth/}{distributions.tex}

\section{Предельные теоремы}

\import{sections/probth/}{limit_thms.tex}

\section{Независимые случайные величины и закон больших чисел}

\import{sections/probth/}{lln.tex}

\section{Геометрическая вероятность}

\import{sections/probth/}{geom_probty.tex}

\section{Колмогоровская аксиоматика}

\import{sections/probth/}{kolmogorov_axioms.tex}

\section{Случайные векторы}

\import{sections/probth/}{rand_vectors.tex}

\section{Условное математическое ожидание}

\import{sections/probth/}{conditional_exp.tex}

\section{Условное распределение}

\import{sections/probth/}{conditional_distr.tex}

\section{Марковские цепи}

\import{sections/probth/}{markov_chain.tex}

\section{Олимпиадные задачи по теории вероятностей}

\import{sections/probth/}{kolmogorov_olimpyad.tex}

\chapter{Математическая статистика}

\textbf{Основная задача статистики} --- по данной выборке $(X_1\ldots X_n)$ сделать выводы о распределении $P_X$.

\section{Структура курса}

\section{Виды сходимостей случайных векторов}

\import{sections/stat/}{vec_convergence.tex}

\section{Параметрическая модель. Статистики и оценки}

\import{sections/stat/}{parametric_models.tex}

\section{Методы нахождения оценок}

\section{Сравнение оценок}

\section{Оптимальные оценки}

\section{Байесовские оценки}

\section{Доверительные интервалы}

\section{Линейная регрессия}

\import{sections/stat/}{linear_regression.tex}

\section{Проверка статистических гипотез}

\end{document}