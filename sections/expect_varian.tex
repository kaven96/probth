\subsection{Математическое ожидание}

Математическое ожидание даёт оценку случайной величины <<в среднем>>, что часто бывает нужно в реальных задачах.

\begin{definition}[Математическое ожидание]
    Пусть $x_1, x_2, \ldots, x_n, \ldots$ --- возможные значения случайной величины $\xi$, а $p_1, p_2, \ldots, p_n, \ldots$ --- соответствующие им вероятности. \\
    Если ряд     
    \begin{equation}
        \mathbb{E} \xi = \sum_{n=1}^\infty x_n p_n   
    \end{equation}
    
    сходится абсолютно, то его сумма называется \textit{математическим ожиданием дискретной случайной величины}.

    Для непрерывных с.в.:
    \begin{equation}
        \mathbb{E}\xi = \int x \cdot p(x) dx = \int x \cdot d F(x)
    \end{equation}

    \textbf{Мат. ожидание линейно:}

    \begin{equation}
        \mathbb{E}(c_1 \xi_1 + c_2 \xi_2) = c_1 \mathbb{E} \xi_1 + c_2 \mathbb{E} \xi_2 
    \end{equation}
\end{definition}

\subsection{Неравенство Маркова}

\begin{theorem}[(Маркова)]
    Пусть $\xi$ принимает только неотрицательные значения. Пусть $a > 0$, тогда

\begin{equation}
    P(\xi \geq a) \leq \frac{\mathbb{E}\xi}{a}
\end{equation}
\end{theorem}

\subsection{Дисперсия}

Важное значение имеет то, насколько сильно случайные величины могут отличаться от своего математического ожидания. Для такой оценки можно использовать дисперсию и некоторые другие числовые характеристики.

\begin{definition}[Дисперсия]
    Дисперсия случайной величины:
    \begin{equation}
        \mathbb{D} \xi := \mathbb{E}(\xi - \mathbb{E})^2 = \mathbb{E} \xi^2 - (\mathbb{E}\xi)^2
    \end{equation}
\end{definition}

\subsection{Неравенство Чебышёва}

\begin{theorem}[Неравенство Чебышёва]
    Пусть $\eta$ --- любая случайная величина. Пусть $b>0$. Тогда:
    \begin{equation}
        P(|\eta - \mathbb{E} \eta| \geq b) \leq \frac{\mathbb{D}\eta}{b^2}    
    \end{equation}
\end{theorem}

\newpage