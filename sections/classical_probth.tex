\marginpar{Если идти последовательно, как в лекции и книгах, то этот раздел должен присутствовать. Но нужен ли он в конспекте??}

Для классической вероятности характерны следующие свойства:
\begin{enumerate}
    \item исходы образуют полную группу событий, т.е. хотя бы один из возможных элементарных исходов произойдет;
    \item события попарно несовместны --- может произойти только одно из них, например, может выпасть только 1 из граней кубика;
    \item все исходы равновероятны.
\end{enumerate}

\subsection{Случайное событие, вероятность}

\begin{definition}[Вероятность]
    Вероятность --- отношение числа благоприятствующих исследуемому событию исходов к числу всех возможных исходов.
\end{definition}

\begin{definition}[Событие]
    Событие A --- подмножество множества элементарных исходов, из которых складывается искомое.
\end{definition}

\begin{definition}[Пространство элементарных исходов]
    Пространство элементарных исходов $\Omega$ --- пространство всех возможных исходов эксперимента.
\end{definition}

\textbf{Свойства вероятности}:
\begin{enumerate}
    \item $P(\Omega) = 1$ --- вероятность того, что произойдет любой из возможных исходов равна 1 по св-ву классической вероятности;
    \item $P(\emptyset) = 0$, $P(A) = 0 \Leftrightarrow  A = \emptyset$;
    \item $P(A \sqcup B) = P(A) + P(B)$ --- сигма-аддитивность, распространяется и на случай конечного объединения событий;
    \item $P(A \cup B) = P(A) + B(B) - P(A \cup B)$;
    \item $P(A_1 \cup A_2 \cup \ldots A_n) = P(A_1) + P(A_2) + \ldots + P(A_n) - P(A_1 \cap A_2) - \ldots - P(A_{n-1} \cap A_n) + \ldots + (-1)^{n-1} P(A_1 \cap \ldots \cap A_n)$ --- формула включений-исключений;
    \item $\bar{A} := \Omega \setminus A$; $P(\bar{A}) = 1 - P(A)$ --- отрицание события; 
    \item $P(A_1 \cup A_2 \cup \ldots A_n) \leq P(A_1) + \ldots + P(A_n)$;
    \item для $A_{i+1} \subseteq A_i$ таких, что $\bigcap_{i=1}^{\infty} A_i = \emptyset$, то $\lim_{k \to \infty} P(A_i) = 0$ --- свойство непрерывности. 
\end{enumerate}

\subsection{Условная вероятность}
 
\begin{definition}[Условная вероятность]
    Условная вероятность --- это вероятность того, что событие A произойдет, при условии, что событие B уже произошло. \\
    Обозначается $P(A|B)$ --- <<вероятность A при условии B>>. \\
    \begin{equation}
        P(A|B) = \frac{|A\cup B|}{|B|} = \frac{P(A \cup B)}{P(B)}
    \end{equation}
\end{definition}

\subsubsection{Теорема умножения и независимость событий}

Теорема умножения позволяет избавиться от нуля в знаменателе, в формуле условной вероятности.

\begin{equation}
    P(A|B) \cdot P(B) = P(A \cup B)
\end{equation}

Событие A не зависит от события B, если $P(A|B) = P(A)$, т.е. наступление события B никак не влияет на вероятность наступления события A. \\
Если событие A независимо от события B, то имеет место равенство:

\begin{equation}
    P(A) \cdot P(B|A) = P(B) \cdot P(A)
\end{equation}

Отсюда, при $P(A)>0$ находим, что 

\begin{equation}
    P(A|B) = P(B)
\end{equation}

т.е. событие B также независимо от A. Т.о. свойство независимости событий взаимно.

События $A_1, \ldots, A_n$ могут быть независимыми в совокупности и попарно. 

\subsubsection{Формула полной вероятности}

Разделим множество $\Omega$ на непересекающиеся подмножества $B_1, \ldots, B_k$.

\subsubsection{Формула Байеса}

\newpage