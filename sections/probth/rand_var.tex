\subsection{Вероятностные пространства в классическом случае и в схеме испытаний Бернулли}

Обобщим вероятностные пространства для классической вероятности и схемы испытаний Бернулли. \\
Вероятность элементарного исхода $\omega_i$ будем обозначать через $p_i$. \\
Очевидно, что $p_i \in [0,1]$, и $p_1 + \ldots + p_n = 1$. \\
Остальные свойства вероятности так же выполнены для обобщенного понятия вероятности.

% Please add the following required packages to your document preamble:
% \usepackage{graphicx}
\begin{table}[h]
    \centering
    \resizebox{\textwidth}{!}{%
    \begin{tabular}{|l|l|l|l|}
    \hline
    \multicolumn{1}{|c|}{Понятие}                                                  & \multicolumn{1}{c|}{Классическая вероятность} & \multicolumn{1}{c|}{Сх. испытаний Бернулли}                          & \multicolumn{1}{c|}{Обобщение}            \\ \hline
    \begin{tabular}[c]{@{}l@{}}Пространство\\ элементарных \\ исходов\end{tabular} & $\Omega = \{\omega_1, \ldots, \omega_n\}$     & $\Omega = \{\omega_1, \ldots, \omega_{2^n}\}$                        & $\Omega = \{\omega_1, \ldots, \omega_n\}$ \\ \hline
    \begin{tabular}[c]{@{}l@{}}Элементарный \\ исход\end{tabular}                  & $\omega_i$ -- 1 из возможных исходов          & $\omega_i = (x_1, \ldots, x_n)$, где $x_i=\{0,1\}$                   & $\omega_i$ -- элементарный исход          \\ \hline
    \begin{tabular}[c]{@{}l@{}}Вероятность\\ элементарного \\ исхода\end{tabular}  & $P(\omega_i) = \frac{1}{n}$                   & $P(\omega_i) = p^{\sum_{i=1}^n x_i}\cdot (1-p)^{n-\sum_{i=1}^n x_i}$ & $P(\omega_i) = p_i$, где $p_i \in [0,1]$  \\ \hline
    \begin{tabular}[c]{@{}l@{}}Вероятность \\ события\end{tabular}                 & $P(A) = \frac{|A|}{n}$                        & $P(A) = \sum_{\omega_i \in A} P(\omega_i)$                           & $P(A) = \sum_{\omega_i \in A} p_i$        \\ \hline
    \end{tabular}%
    }
\end{table}

\subsection{Определение случайной величины}

Пусть дано Пространство элементарных исходов $\Omega$, и вероятностная мера $P$ на нём.

\begin{definition}[Случайная величина]
    Случайной величиной называется любая функция $\xi \; : \; \Omega \rightarrow \mathbb{R}$. \\
    Например, для игральной кости $\Omega = \{1,2,3,4,5,6\}$, зададим $\xi$ как $\xi(\omega) = \omega^2$.
\end{definition}

Случайная величина ставит в соответствие каждому элементарному исходу какое-либо конкретное число из множества действительных чисел. \\
Таким образом, "выпадение" какого-то из исходов случайно, но значение случайной величины уже конкретно.

\textbf{Пример.} $\xi =$ $\#$ треугольников в случайном графе. \\
Количество событий, когда на имеющихся вершинах появился треугольник случайно, но, зная количество таких событий, мы можем получить конкретное $\xi$.

\subsection{Функция распределения случайной величины}

Существует важная задача создания методов изучения случайных величин. \\
Случайная величина под влиянием случайных обстоятельств может принимать различные значения. Заранее предсказать, какое значение примет величина невозможно, т.к. оно меняется случайным образом от испытания к испытанию. \\
Для того, чтобы задавать вероятности значений случайных величин вводится понятие \textbf{функции распределения случайной величины}.

\begin{definition}[Распределение случайной величины]
    Распределением случайной величины называется вероятность того, что случайная величина $\xi$ принимает конкретное значение $y_i$:
    \begin{equation}
        P(\xi = y_i) = P(\{\omega_j \; : \; \xi(\omega_j) = y_i\})
    \end{equation}
Т. е. это вероятность всех элементарных исходов, при которых случайная величина принимает значение $y_i$.
\end{definition}

\begin{definition}[Функция распределения случайной величины]
Пусть $\xi$ --- случайная величина, $x \in \mathbb{R}$ --- произвольное действительное число. Вероятность того, что $\xi$ примет значение, меньшее, чем $x$ называется \textit{функцией распределения случайной величины}.

\begin{equation}
    F_\xi (x) := P(\xi \leq x) = \sum_{i \; : \; y_i \leq x} P(\xi=y_i)
\end{equation}

\end{definition}

\subsection{Абсолютно непрерывные случайные величины}

Для а.н.с.в. характерна непрерывная функция распределения. Пусть $p(x)$ --- полотность распределения, тогда она обладает свойствами:
\begin{enumerate}
    \item $\forall x \; p(x) \geq 0$;
    \item $\int_{-\infty}^{+\infty} p(x) d(x) =1$;
\end{enumerate}

В а.н. случае функция распределения представляется в виде:
\begin{equation}
    F_\xi(x) = \int_{-\infty}^x p(t) dt
\end{equation}

\subsection{Сингулярные распределения}

\newpage